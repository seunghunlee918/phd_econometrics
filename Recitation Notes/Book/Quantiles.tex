

%%%%%%%%%%%%%%%%%%
\chapter{Quantile Regression}
\section{Introduction: A Check Function Approach}
The usual linear regression, which has the form
\[
y= X\beta+u
\]
where $E(Xu)=0$ (or more restrictively, $E(u|X)=0$) is aimed at capturing the conditional mean of $y$ given $X$ at a certain value. However, there is no reason to restrict our attention to just a conditional mean. In fact, we an do more. For instance, what is the conditional median? What about the top 10\%, or 1\%? \textbf{Quantile regression} aims to capture different values of $\beta$ depending on the location of the conditional distribution. \par 
Rigorously speaking, the quantile regression seeks to estimate the conditional quantile
\[
q_\tau(y | X)=X\beta_\tau
\]
where $\tau\in[0,1]$ is the percentile of our choice satisfying $F_{y|X}(X\beta_\tau|X)=\tau$. To characterize the conditional quantile function, we need to use the \textbf{check function}. Since we have
\[
F_{y|X}(X\beta_\tau|X)=\Pr(y\leq X\beta_\tau|X)=\tau
\]
we can write as
\[
\tau-\Pr(y\leq X\beta_\tau|X)=0
\]
Since $\Pr(y\leq X\beta_\tau|X)$ is equal to $E[1(y- X\beta_\tau\leq 0)]=E[1(u\leq0)]$, we can obtain the condition
\[
E[\tau-1(y- X\beta_\tau\leq 0)|X]=0
\]
and using the law of iterated expectations, this also implies $E[(\tau-1(y- X\beta_\tau\leq 0))X]=0$. The check function can be defined as
\[
\rho_\tau(u)=u(\tau-1(u\leq0))
\]
\section{Usage of Check Function}
To see what this check function is, I will talk about two specific cases
\begin{itemize}
\item Median: Let $\tau=1/2$. Then the check function becomes
\[
\rho_{1/2}(u)=\begin{cases}-\frac{1}{2}u & (u\leq 0) \\ \frac{1}{2}u & (u>0) \end{cases} = \frac{1}{2}|u|=\frac{1}{2}|y-X\beta_{1/2}|
\]
This becomes equivalent to solving the least absolute deviation problem.
\item $\tau=1/3$: Then the check function becomes
\[
\rho_{1/3}(u)=\begin{cases}-\frac{2}{3}u & (u\leq 0) \\ \frac{1}{3}u & (u>0) \end{cases}
\]
which has a kink at $u=0$ and is asymmetric.\par
\end{itemize}\par
For all values of $\tau$ the check function has a kink at $u=0$. However, the check function is convex and can be shown to be a lipschitz function. 
\begin{mdframed}[backgroundcolor=green!5] 
\begin{theorem}[Convex Functions are Lipschitz Functions?] Let $f$ be a convex function and $K$ be a closed, bounded set contained in the relative interior of the domain of $f$. Then $f$ is Lipschitz continuous on $K$. 
\begin{proof}
Suppose that $f$ is convex on $[a,b]$. Select $c,d$ s.t. $a<c<d<b$ and let $t_1\in[d,b]$, $t_2\in[a,c]$ and $x_1,x_2\in[c,d]$. Since $f$ is convex we can write
\[
\frac{f(c)-f(t_2)}{c-t_2}\leq \frac{f(x_2)-f(x_1)}{x_2-x_1}\leq \frac{f(t_1)-f(d)}{t_1-d}
\]
Then, let $L=\max\left[\left|\frac{f(c)-f(t_2)}{c-t_2}\right|,\left|\frac{f(t_1)-f(d)}{t_1-d}\right|\right]$, we then have 
\[|f(x_2)-f(x_1)|\leq L|x_2-x_1|\]
\end{proof}
\end{theorem}
\end{mdframed}
\par
\section{Obtaining Quantile Regression Estimator}
The minimization problem that should be solved in quantile regression is
\[
\min_\beta E[ \rho_\tau(y-X\beta_\tau)|X]
\]
which can be written as
\begin{align*}
E[ \rho_\tau(y-X_i\beta_\tau)|X] &= (\tau-1)\int_{-\infty}^{a} (y-a)f_{Y|X}(y|x)dy+ \tau\int_{a}^\infty(y-a) f_{Y|X}(y|x)dy\\
\end{align*}
where $a=X\beta_\tau$. Take the first order condition w.r.t. $a$ to get
\begin{gather*}
-(\tau-1)\int_{-\infty}^a f_{Y|X}(y|x)dy - \tau\int_a^\infty f_{Y|X}(y|x)dy=0\\
\iff -\tau + F(a|X)=0
\end{gather*}
Thus, the $\beta_\tau$ that solves
\[
X\beta_\tau=a=F^{-1}_{Y|X}(\tau|X)
\]
is the $\beta_\tau$ that we are looking for. We can also solve for
\[
\min_\beta E[\rho_\tau(y-X\beta_\tau)X]
\] 
or in a sample analogue
\[
\min_\beta \frac{1}{n}\sum_{i=1}^n \rho_\tau(y_i-X_i\beta_\tau)X_i
\]
to find the quantile estimator $\beta_\tau$. For suitable conditions, the resulting estimator is CAN. 
\begin{mdframed}[backgroundcolor=yellow!5] 
\begin{example}[Levin, 2001]
As part of the education production function literature, the effect of class size on various outcomes for the student is controversial (Lazear, 2001). This paper, while controlling for a large number of observable characteristics and endogeneity in class size variable, estimates education production function using a quantile regression. In doing so, it attempts to analyze the conventional claims that reducing class size improves learning directly via and increased allocation of teaching per student. While it does not find statistically significant impact of class size once peer effect is controlled for, peer effect has a positive and significant effect for those in the lower portion of the achievement distribution for math and language scores. 
\end{example}

\begin{example}[Autor, Houseman, \& Kerr 2017\footnote{Technically, this paper uses Chernozhukov-Hansen Instumental Variables Quantile Regression, which is not part of our curriculum yet. You may learn this in Microeconometrics course next year.}]
Using the Detroit's welfare-to-work program, this paper studies the the effect of two government employment programs - direct hire assistance and temporary-help job placements - on distribution of participant's earnings over a 7-quarter period. The paper finds that for the low-tail of the earnings distribution, neither programs are effective. The direct-hire increases earnings for the high-tail but temporary-help placements negatively affect the distribution for the same group.  Autor and Houseman (2010) have studied the same program 7 years ago without using the quantile approach and find that on average, the same program lead to earnings gain. The key takeaway is that by using quantile regression, you can unmask the effect at a different quantile that you cannot find out through conditional expectations. 
\end{example}
\end{mdframed}

