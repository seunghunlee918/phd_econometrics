
\chapter{Treatment Effects: Random Assignment and Estimation under Unconfoundedness}
\section{Introduction to Treatment Effects: Causality and Treatment Assignment}
Now, our interest is in finding out whether some $X$ variable \textbf{causes} $Y$, not just whether they are correlated. Generally, correlation is not enough to argue causation. Suppose that you find that $cov(X,Y)\neq0$. This can happen because
\begin{itemize}
\item $X$ do cause $Y$, which is good for us. But..
\item $Y$ could also cause $X$. So there is a reverse causality bias here
\item $Z$ mutually affects $X$ and $Y$. This is an omitted variable bias and leads to nonzero correlation even if $X$ and $Y$ has no connection whatsoever. 
\end{itemize} \par
In any study examining program evaluation or quasi-experimental setting, the key issue is the assignment of the treatment, which is considered binary for the moment. There are three possible cases, which are
\begin{itemize}
\item \textbf{Random Assignment}: This would be a case when we can guarantee that the assignment to the treatment arms (treatment and control) are determined by chance - like flipping a coin or random draw of numbers. This rarely happens in social science, as people can voluntarily opt-in and opt-out of treatment. 
\item \textbf{Selection on Observables}: The treatment assignment is effectively random once we condition on some observable covariates. This is also known as ignorability or unconfoundedness assumptions.
\item \textbf{Selection on Unobservables}: The assignment depends fundamentally on unobservables, or in other words, we cannot break down the dependence structure of assignment using observed variables. For instance, even in a clinical trials that involve voluntary participation, it is very likely that (lack of) risk averseness determines participation. How do we control for risk averseness?
\end{itemize}
\section{Theoretical Framework for Treatment Effect}
We will consider a binary treatment variable - whether individual $i$ received a treatment or not (in the treatment group or control group). Define a variable $D_i$ s.t.

\[
D_i = \begin{cases} 1 & \text{If treated} \\ 0 & \text{If not treated}\end{cases}
\]
In the above notation, $i$ indexes the unit of the treatment - an individual, household,  county, firm, and etc. For each unit $i$, there are two possible outcomes. One is the outcome without treatment,  $Y_i(0)$. The other is the outcome with treatment, $Y_i(1)$. This can be seen as a counterfactual framework in the sense that if we get $Y_i(1)$ for unit $i$, we cannot get $Y_i(0)$ and vice versa. We always have a missing data problem in this regard. A useful way of writing this is
\[
Y_i = D_iY_i(1) + (1-D_i)Y_i(0)
\]
The left hand side is an outcome for unit $i$, treated or untreated. \textit{This is observed for every $i$}. The right hand side is meant to capture that the observed outcome for individual $i$ is either \textit{one} of $Y_i(1)$ or $Y_i(0)$. This framework is sometimes called \textbf{potential outcome framework} and will come in handy when we analyze various treatment effect results. \par
We are interested in how an outcome for unit $i$ changes between treatment and control. In other words,
\[
TE_i = Y_i(1)-Y_i(0)
\]
Another parameter of interest could be the treatment effect averaged over those who share the common covariate value, which is
\[
TE(x) = E(TE_i|X_i=x)
\]
We could also be interested in the treatment effect averaged over the population. This is the average treatment effect, defined as
\[
ATE=E(TE_i)=E(Y_i(1)-Y_i(0))
\]
Of course there are more, most of which we will learn over the course of this sequence. There are average treatment effect on the treated, ATE for the untreated, etc. \par
Because of the fundamental  problem of a missing data, we are forced to make an assumption about the things we cannot observe. Take the ATE for example. We can write
\begin{align*}
E[Y_i(1)-Y_i(0)] & = E[Y_i(1)]-E[Y_i(0)]\\
&=\{\Pr(D_i=1)E[Y_i(1)|D_i=1]+(1-\Pr(D_i=1))E[Y_i(1)|D_i=0]\}\\
&-\{\Pr(D_i=1)E[Y_i(0)|D_i=1]+(1-\Pr(D_i=1))E[Y_i(0)|D_i=0]\}
\end{align*}
Here is the problem: We can get what $E[Y_i(1)|D_i=1]$ and $E[Y_i(0)|D_i=0]$ are from the data. This is because \footnote{You might have noticed that I rarely tag equations in my recitation notes. So when I do, it means I am going back to this very often. }
\begin{gather*}
E[Y_i|D_i=1]=E[1\cdot Y_i(1)-(1-1)\cdot Y_i(0)|D_i=1]=E[Y_i(1)|D_i=1]\\
E[Y_i|D_i=0]=E[0\cdot Y_i(1)-(1-0)\cdot Y_i(0)|D_i=0]=E[Y_i(0)|D_i=0] \tag{\text{TE}}\label{TE}
\end{gather*}
So we can get $E[Y_i(1)|D_i=1]$ and $E[Y_i(0)|D_i=0]$ from the data - take the expected value of observed $Y_i$ conditional on $D_i=1 \ \text{and}\ 0$. We cannot do the same for $E[Y_i(1)|D_i=0]$ and $E[Y_i(1)|D_i=0]$. This forces us to make an assumption about what the two unobserved values could be. \par
Before going on, it would be useful to characterize the counterfactual outcome in a more econometrics-friendly context. Define the counterfactual outcomes as
\[
Y_i(D_i=d)=\mu(X_i,d)+\epsilon_i(d)
\]
where $d$ can take either $0, 1$. What we want to learn involves understanding the joint distribution of the variables $Y_i(0)$ and $Y_i(1)$. Take the treatment effect at $X_i=x$, written as
\[
TE(x)=\mu(x,1)-\mu(x.0)
\]
This is the effect of the treatment on the outcome $Y_i$ for $X_i=x$. The interpretation is that if we shift everyone with $X_i=x$ from the control group to the treatment group, the average outcome increases by $TE(x)$. We can also calculate ATE as
\begin{align*}
ATE&=E[Y_i(1)-Y_i(0)]=E[\mu(X_i,1)-\mu(X_i,0)+\epsilon_i(1)-\epsilon_i(0)]\\
&=E\left[E[\mu(X_i,1)-\mu(X_i,0)+\epsilon_i(1)-\epsilon_i(0)|X_i]\right]=E\left[E[\mu(X_i,1)-\mu(X_i,0)|X_i]\right]\\
&=E[\mu(X_i,1)-\mu(X_i,0)]=E[TE(X_i)]
\end{align*}
\section{Randomized Assignment}
We are back to having to assume what $E[Y_i(1)|D_i=0]$ and $E[Y_i(1)|D_i=0]$ are. A \textbf{random assignment} assumes that the outcome is independent of the treatment status. More formally
\[
(Y_i(0), Y_i(1))\perp \!\!\! \perp D_i \ \tag{\text{RA}}\label{RA}
\]
This implies that
\[
E[Y_i(1)]=E[Y_i(1)|D_i=1]=E[Y_i(1)|D_i=0]
\]
and similarly for $E[Y_i(0)]$. If we go back to the (\ref{TE}) equation, the problem was that we were unable to make any statement about $E[Y_i(1)|D_i=0]$ and $E[Y_i(1)|D_i=0]$. Now what we are doing is to equate
\begin{gather*}
E[Y_i|D_i=1]=E[Y_i(1)|D_i=1]=E[Y_i(1)|D_i=0]\\
 E[Y_i|D_i=0]=E[Y_i(0)|D_i=0]=E[Y_i(0)|D_i=1]
\end{gather*}
and effectively make the statements about the previously unobserved values. This allows us to rewrite the ATE as 
\begin{align*}
E[Y_i(1)-Y_i(0)]&=E[Y_i(1)]-E[Y_i(0)]\\
&=E[Y_i(1)|D_i=1]-E[Y_i(0)|D_i=0] \ (\because \ref{RA})\\
&=E[Y_i|D_i=1]-E[Y_i|D_i=0] \ (\because \ref{TE})
\end{align*}
Therefore, we can estimate ATE by mapping $E[Y_i(1)]$ to $E[Y_i|D_i=1]$, $E[Y_i(0)]$ to $E[Y_i|D_i=0]$. 
\section{Conditional Independence Assumption}
In this setup, we are assuming that conditional on $X_i$, the outcomes and $D_i$ are independent. Formally, we can write
\[
 (Y_i(0), Y_i(1)) \perp \!\!\! \perp D_i|X_i \ \tag{\text{CIA}}\label{CIA}
\]
We can alternatively conceptualize this framework in the following way: Define $D_i$ as
\[
D_i = 1(u_i<p(X_i))
\]
where $u_i\equiv U[0,1]$ determines selection and $p(X_i)$ can be interpreted as a propensity score - how much are you likely to be treated given that you have covariates specified as some $X_i$. This framework, along with the (\ref{CIA}), allows us to say the following:
\footnotesize{\begin{align*}
E[Y_i(1)|X_i=x]&=E[Y_i(1)|D_i=1,X_i=x]=E[Y_i(1)|D_i=0,X_i=x] \ (\because \ref{CIA})\\
&\implies E[\mu(x,1)+\epsilon_i(1)|D_i=1,X_i=x]=E[\mu(x,1)+\epsilon_i(1)|D_i=0,X_i=x]\\
&\implies E[\mu(x,1)|D_i=1,X_i=x]-E[\epsilon_i(1)|D_i=1,X_i=x]\\
&=E[\mu(x,1)|D_i=0,X_i=x]+E[\epsilon_i(1)|D_i=0,X_i=x]\\
&\implies E[\epsilon_i(1)|D_i=1,X_i=x]=E[\epsilon_i(1)|D_i=0,X_i=x]\ (\because \mu(x,1)\text{ is same for both cases})\\
&\implies E[\epsilon_i(1)|u_i<p(x),X_i=x]=E[\epsilon_i(1)|u_i>p(x),X_i=x]\\
&\implies (\epsilon_i(1), \epsilon_i(0))\perp \!\!\! \perp u_i|X_i \tag{\text{CIA2}}\label{CIA2}
\end{align*}}\normalsize
Thus, conditional on observed values for the covariate $X_i$, the unobserved elements of the selection rule $u_i$ are independent of the unobserved elements of the error process $(\epsilon_i(1), \epsilon_i(0))$.\par
 The caveat, however, is that $p(x)\in(0,1)$. If $p(x)=1$, then for every possible $u_i$, $D_i=1$ - everyone gets treated. Therefore, there is no meaningful statement we can make about the $D_i=0$ case. We can say similarly for $p(x)=0$ case. In some textbooks, this is also known as \textbf{overlap} assumption - each unit in the defined population should have some chance of being treated and some chance of being in the control group. \footnote{If both (\ref{CIA}) and overlap is satisfied, we say that have \textbf{strong ignorability} (Rosenbaum, Rubin 1983).} \par
Under CIA, we can derive the treatment effect as follows
\footnotesize{\begin{align*}
E[Y_i|D_i=1, X_i=x]-E[Y_i|D_i=0, X_i=x]&=E[\mu(x,1)+\epsilon_i(1)|D_i=1, X_i=x]-E[\mu(x,0)+\epsilon_i(0)|D_i=0, X_i=x]\\
&=E[\mu(x,1)|D_i=1, X_i=x]+E[\epsilon_i(1)|D_i=1, X_i=x]\\
&-E[\mu(x,0)|D_i=0, X_i=x]-E[\epsilon_i(0)|D_i=0, X_i=x]\\
&=\mu(x,1)+E[\epsilon_i(1)|X_i=x]-\mu(x,0)-E[\epsilon_i(0)|X_i=x] \ (\because \ref{CIA2})\\
&=\mu(x,1)-\mu(x,0)
\end{align*}}\normalsize
Note that 
\footnotesize{\begin{align*}
E[Y_i(1)-Y_i(0)|X_i=x] &= E[Y_i(1)|X_i=x]- E[Y_i(0)|X_i=x]\\
&=(\Pr(D=1|X_i=x)\cdot E[Y_i(1)|D_i=1,X_i=x]+(1-\Pr(D=1|X_i=x))E[Y_i(1)|D_i=0,X_i=x])\\
&-(\Pr(D=1|X_i=x)\cdot E[Y_i(0)|D_i=1,X_i=x]+(1-\Pr(D=1|X_i=x))E[Y_i(0)|D_i=0,X_i=x])\\
&=  E[Y_i(1)|D_i=1,X_i=x]- E[Y_i(0)|D_i=0,X_i=x] \ (\because \ref{CIA}) \\
&=  E[Y_i|D_i=1, X_i=x]-E[Y_i|D_i=0, X_i=x]
\end{align*}}\normalsize
Thus, under (\ref{CIA}) we can back out the ATE for $X_i=x$ using the observables. This assumption allows us to map $ E[Y_i(1)|X_i=x]$ to $ E[Y_i|D_i=1, X_i=x]$ and $E[Y_i(0)|X_i=x]$ to $ E[Y_i|D_i=0, X_i=x]$, which is impossible without it. \par
This is called a \textbf{regression adjustment} method. The treatment effect for $X_i=x$ using a regression adjustment can be obtained by utilizing the fact that 
\[
\mu(x,1)=E[Y_i|D_i=1, X_i=x], \ \mu(x,0)=E[Y_i|D_i=0, X_i=x]
\]
and regressing on the subsample of each treatment arm to get $\hat{\mu}(x,1)$ and $\hat{\mu}(x,0)$. Thus
\[
\frac{1}{N}\sum_{i=1}^N(\hat{\mu}(x,1)-\hat{\mu}(x,0))
\]
\section{Inverse Probability Weighting}
We can obtain the average treatment effect using a different approach. This is an \textbf{inverse probability weighting}. The steps are as follows
\begin{enumerate}
\item Estimate the propensity score $p(X_i)$ by computing
\[
\hat{p}_n(X_i)=\Pr(D_i=1|X_i=x)
\]
\item Use the magic formula to estimate $E(TE(x)|x\in A)$: That is,
\[
\frac{\sum_{D_i=1,x_i\in A}a_iY_i}{\sum_{D_i=1,x_i\in A}a_i}- \frac{\sum_{D_i=0,x_i\in A}b_iY_i}{\sum_{D_i=0,x_i\in A}b_i}
\]
where $a_i=\frac{1}{\hat{p}_n(x_i)}, b_i=\frac{1}{1-\hat{p}_n(x_i)}$. The above can also be written as
\[
\frac{\sum_{i=1,x_i\in A}^N\frac{D_iY_i}{\hat{p}(X_i)}}{\sum_{i=1,x_i\in A}^N\frac{D_i}{\hat{p}(X_i)}}-\frac{\sum_{i=1,x_i\in A}^N\frac{(1-D_i)Y_i}{1-\hat{p}(X_i)}}{\sum_{i=1,x_i\in A}^N\frac{1-D_i}{1-\hat{p}(X_i)}}\]
\end{enumerate}\par
The reason this works is as follows: Notice that
\begin{align*}
D_iY_i& = D_i(D_iY_i(1)+(1-D_i)Y_i(0))=D_iY_i(1)\\
(1-D_i)Y_i&=(1-D_i)Y_i(0)
\end{align*}
Thus, we have
\begin{align*}
E\left[\frac{D_iY_i}{p(X_i)}\right]&=E\left[E\left[\frac{D_iY_i(1)}{p(X_i)}|X_i\right]\right]\\
&=E\left[\frac{E[D_i|X_i]E[Y_i(1)|X_i]}{p(X_i)}\right]\\
&=E\left[\frac{p(X_i)E[Y_i(1)|X_i]}{p(X_i)}\right]=E[E[Y_i(1)|X_i]]=E[Y_i(1)]
\end{align*}
so we have $E\left[\frac{D_iY_i}{p(X_i)}\right]=E[Y_i(1)], E\left[\frac{D_iY_i}{p(X_i)}|X_i\right]=E[Y_i(1)|X_i]$. We can make the similar arguments for $Y_i(0)$ and get $E\left[\frac{(1-D_i)Y_i}{1-p(X_i)}\right]=E[Y_i(0)], E\left[\frac{(1-D_i)Y_i}{1-p(X_i)}|X_i\right]=E[Y_i(0)|X_i]$. Thus, the ATE for $X_i\in A$ can be written as
\[
\frac{1}{N}\sum_{i=1}^N\left(\frac{D_i Y_i}{p(X_i)}-\frac{(1-D_i)Y_i}{1-p(X_i)}\right) \ \forall X_i\in A
\]
In most cases, the propensity score is estimated. So we use the estimator involving the $\hat{p}$, which is what we have in the lecture notes. It is said that normalizing the weights to one in finite samples improves the mean squared error properties of the estimator (Imbens, Rubin (2019) - \textit{Causal Inference for Statistics, Social, and Biomedical Sciences: An Introduction})
\section{Matching Estimation}
The idea behind matching estimation is to impute the values for something we cannot get from the data: Namely,  $Y_i(0)$ for those in $D_i=1$ and $Y_i(1)$ for those in $D_i=0$. In other words, you try to find a `close' match for a particular unit $i$ in a different treatment arm. When we say we find $k$-closest neighbors for unit $i$ in $D_i=0$, we find $k$ individuals in $D_i=1$ that has close traits ($X_i$) to individual $i$, or $k$ individuals with the smallest values of $||x_j - x_i||$. Then, we construct a counterfactual $Y_i(1)$ by taking a (weighted) average over the $Y_i$'s of the $k$ individuals found in the other group. The treatment effect would than be $Y_i(1)$ -$Y_i$, where $Y_i(1)$ is calculated as in previous sentence. \par
Let's try with $k=1$ - we find the one individual in the opposite treatment arm that has the similar value of $X_i$. Then, the average treatment effect can be written as
\[
\frac{1}{N}\sum_{i=1}^N(\hat{Y}_i(1)-\hat{Y}_i(0))
\]
where
\[
\hat{Y}_i(1)=\begin{cases} Y_i & (D_i=1) \\ \text{imputed value} & (D_i=0)\end{cases}, \ \hat{Y}_i(0)=\begin{cases} \text{imputed value} & (D_i=1) \\ Y_i & (D_i=0)\end{cases}
\]
\par
There are some caveat in this approach
\begin{itemize}
\item The covariate $X_i$ that is used to find the closest match in the other treatment arm should not affect the assignment of the treatment. In other words, if the assignment of treatment vs. control depends on some subset of $X_i$, you should not use that subset as a standard for finding the closest match.
\item The overlap condition becomes critical. If it is not satisfied, i.e. for some $X_i$, $p(X_i)=1$ or $0$, then we are unable to find a closest match in the other treatment arm because for individuals with that covariate value, all of them are either in control or treatment group and not spread around. 
\item Who are the neighbors? The answer might depend on what distance measure we use. Moreover, how many of those in the treatment can be considered neighbors? There is a trade-off in the sense that using larger $k$ would force us to put someone who is not `close' as neighbors and using small $k$ may cause difficulty in imputing counterfactual values. 
\end{itemize}

\section{Difference-in-Difference Estimation}
This involves a specific framework where we can clearly define a `before and after' - denoted as $t_0$ and $t_1$. No one is treated at $t_0$ but there is a subset of people ($G_i=1$) that are treated at $t_1$. Those in $G_i=0$ are never treated in either time period. If we define
\[
Y_{i,t}=G_iY_{i}(1,t)+(1-G_i)Y_{i}(0,t) \ \ (t\in\{t_0, t_1\})
\]
and impose
\[
Y_i(G_i, t_0)=Y_{i}(t_0) \ \ \text{for both }G_i=1 \ \text{and }G_i=0 
\]
then we will be able to observe $(Y_{i,t_1}, Y_{i,t_0},G_i. X_i)$ for every unit $i$. What we do not observe is $Y_i(1,t_1)$ for those in $G_i=0$ and $Y_i(0,t_1)$ for those in $G_i=1$.  The analogue to the conditional independence assumption in this context is a \textbf{parallel trend assumption}, defined as
\[
(Y_i(1,t_1)-Y_i(t_0), Y_i(0,t_1)-Y_i(t_0)) \perp\!\!\! \perp G_i|X_i
\]\par
To see why they are equal, consider a setting where $D_i=G_i$ and write
\begin{align*}
Y_i = Y_{i,t_1}-Y_{i,t_0}&=D_i(\underbrace{Y_i(1,t_0)-Y_i(1,t_0)}_{Y_i(1)})+(1-D_i)(\underbrace{Y_i(0,t_1)-Y_i(0,t_0)}_{Y_i(0)})\\
&=D_iY_i(1) + (1-D_i)Y_i(0)
\end{align*}
Therefore, we can rewrite the parallel trend assumption as
\[
(Y_i(1), Y_i(0)) \perp\!\!\! \perp D_i|X_i
\]
which is the same as the conditional independence assumption. \par
To apply this in regression, we can write
\[
Y_{it}=\beta_0(X_i)+\beta_1(X_i)\cdot1(t=t_1)+\beta_2(X_i)\cdot G_i + \beta_3(X_i)\cdot G_i\cdot1(t=t_1)+\epsilon_{it}
\]
where $X_i$ is a set of covariates, which can include a constant. In this context, the treatment effect for $X_i=x$ would be
\begin{align*}
TE(x)&=E[Y_i(1)-Y_i(0)|X_i=x]\\
&=E[(Y_i(1,t_1)-Y_i(1,t_0))-(Y_i(0,t_1)-Y_i(0,t_0))|X_i=x]\\
&=x\cdot E\{[(\beta_0+\beta_1+\beta_2+\beta_3)-(\beta_0+\beta_2)]-[(\beta_0+\beta_1)-(\beta_0)]|X_i=x\}\\
&=x\cdot E[(\beta_1+\beta_3)-\beta_1|X_i=x]\\
&=\beta_3 x
\end{align*}
So $\beta_3$ would be our parameter of interest. \par
One difficulty arises from testing the parallel trends assumption. Loosely speaking, what we can do is to select a time period $\tilde{t}<t_0$ and find out if the difference $y_i(t_0)-y_i(\tilde{t})$ is independent with $G_i$. If not independent, we say that there is an autonomous trend. 

%%%%%%%%%%%%%%%


